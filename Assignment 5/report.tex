\documentclass{article}

% Language setting
% Replace `english' with e.g. `spanish' to change the document language
\usepackage[english]{babel}

% Set page size and margins
% Replace `letterpaper' with `a4paper' for UK/EU standard size
\usepackage[letterpaper,top=2cm,bottom=2cm,left=3cm,right=3cm,marginparwidth=1.75cm]{geometry}

% Useful packages
\usepackage{amsmath}
\usepackage{graphicx}
\usepackage[colorlinks=true, allcolors=blue]{hyperref}

\title{Assignment 5 (CSE-4137)}
\author{Raheeb Hassan (Roll: AE-42)}

\begin{document}
\maketitle

\section*{Answer to Problem 1}

We know that, due to how the RSA algorithm is designed
\begin{equation*}
    e_b \cdot d_b \equiv 1 \mod \phi(n)
\end{equation*}
By definition of the modulo operator
\begin{align*}
    e_b \cdot d_b = 1 + \phi(n) \cdot k \\
    \phi(n) \cdot k = e_b \cdot d_b - 1
\end{align*}
So, the required multiple of $\phi(n)$ is $V = e_b \cdot d_b - 1$

\section*{Answer to Problem 2}

As $e_a$ is relatively prime to $V$, using extended euclidean algorithm we can find $d$ such that,
\begin{equation*}
    d \cdot e_a = 1 \mod V
\end{equation*}
That is, we find the inverse of $e_a$ modulo $V$.\\
Now, by definition of the modulo operator
\begin{align*}
    &d \cdot e_a = 1 + V \cdot k'\\
    &d \cdot e_a = 1 + (\phi(n) \cdot k) \cdot k'\\
    &d \cdot e_a = 1 + \phi(n) \cdot k''\\
    &m ^ {d \cdot e_a} = m ^ {1 + \phi(n) \cdot k''} & & \text{($m \in Z_n$)}\\
    &m ^ {d \cdot e_a} = m \cdot {(x ^ {\phi(n)})} ^ {k''}\\ 
    &m ^ {d \cdot e_a} = m & & \text{(Euler's theorem, $m^{\phi(n)} = 1$)}
\end{align*}

Therefore, if we encrypt any message $m$ with secret key $d$, it can be decrypted using $e_a$.
\begin{equation*}
    c ^ {e_a} = (m ^ d) ^ {e_a} = m ^ {d \cdot e_a} = m
\end{equation*}

In other words, Bob can invert Alice's trapdoor permutation and obtain her signature on m (by signing it with $d$)
\begin{equation*}
    \sigma_a = H(m) ^ d
\end{equation*}

\section*{Answer to Problem 3}

If $e_a$ is not relatively prime to $V$, then $gcd(e_a, V) \neq 1$.
Let the prime factorization of V be:
\begin{equation*}
    V = \prod {p_i} ^ {q_i}
\end{equation*}
Now, we define,
\begin{align*}
    &W = \prod {p_i} ^ {q_i} \text{\space where $p_i | e_a$} \\
    &V' = V / W
\end{align*}
As $W$ contains all the multiples of the common factors of $V$ and $e_a$, $gcd(e_a, V') = 1$.\\
Moreover, as $e_a$ and $\phi(n)$ have no common factors, we can be sure that $V'$ is also a multiple of $\phi(n)$.\\\\
Now, we can use $V'$ instead of $V$ and use the same procedure as problem 2.

\end{document}